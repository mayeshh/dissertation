\chapter{Conclusion}
\label{chpt:conclusion}

In the past decade, the advancement of \ac{SPAD} arrays, demonstrating performance compatible with \ac{smFRET} experiments, has opened up numerous exciting opportunities for conducting high-throughput single-molecule fluorescence studies. 
Although there is still room for enhancing detector sensitivity, which has been partially achieved through red-enhanced \ac{SPAD} arrays, and reducing the dark count rate, the current array characteristics, encompassing both sensitivity and the number of \ac{SPAD}s, allow for envisioning several extensions of this research. 
These extensions include equilibrium \ac{HT-smFRET} investigations facilitated by advanced microfluidic formulator devices, \ac{HT-smFRET} kinetics studies employing rapid microfluidic mixers, and high-throughput screening via parallel channel microfluidic lab-on-chip devices (as illustrated in Figure~\ref{fig:microfluidic_examples})~\cite{streets_COB_2014,yeh_SA_2017}.

This fusion of microfluidics and \ac{HT-smFRET} will likely necessitate the development of custom microfluidic designs tailored to leverage the capabilities of the emerging \ac{SPAD} arrays. 
Concurrently, it may drive the innovation of specialized \ac{SPAD} array geometries designed for specific applications. 
Specifically, applications such as rapid microfluidic mixers or parallel channel high-throughput screening would gain advantages from linear \ac{SPAD} arrays with a greater number of \ac{SPAD}s and increased density.

Extending these measurements to incorporate time-resolved detection is not only feasible, as demonstrated earlier, but also highly advantageous. 
Time-resolved detection offers insights into rapidly interconverting sub-populations, which are crucial for understanding dynamic processes occurring on timescales shorter than typical diffusion times. 
Additionally, it aids in identifying transient states, as previously illustrated in Lerner \textit{et al.,} 2018~\cite{lerner_Science_2018}.

With respect to the optical components, multispot excitation techniques employing spatial light modulators, as demonstrated here, might be substituted with more straightforward and cost-effective illumination methods, like the linear illumination technique employed in Ingargiola \textit{et al.,} 2017~\cite{ingargiola_SPIE_2017}. 
This transition away from confocal spot generation, such as in the case of the \ac{LCOS-SLM}s, not only simplifies alignment and promotes broader usage by reducing cost and complexity, but also enhances laser power utilization, reducing both excitation power demands.

Two decades after the initial showcasing of smFRET measurements in solution~\cite{deniz_PNAS_1999}, it is clear that this powerful technique still holds great promise~\cite{lerner_Science_2018}.


\section{Closing remarks}
\label{sec:closing_remarks}

In conclusion, my dissertation work represents a significant step toward achieving the ambitious goal of capturing the dynamic 3D atomic-scale structure of macromolecular machines as they carry out their biological functions. 
Through a combination of cutting-edge techniques and innovative methodologies, we have made substantial progress in this endeavor.

Our approach involves leveraging prior knowledge from multiple existing static structures of stable states and integrating dynamic datasets obtained through non-equilibrium \ac{HT-smFRET} measurements in a microfluidic mixer, utilizing state-of-the-art time-resolved multi-pixel \ac{SPAD} arrays. 
These measurements, conducted on libraries of molecular constructs, enable us to sample multiple inter-atomic distances as a function of reaction time.

The resulting distance distributions serve as distance constraints. 
These constraints, when combined with prior structural information, facilitate large-scale computational energy optimization-based refinement. 
This approach allows us to generate time-resolved and atomically-resolved structures. 
These time-resolved computational structures, coupled with intermediate molecular dynamics simulations, empower us to uncover the 3D atomic-level structure of the macromolecule at each sampled reaction time point.

The ultimate objective of this work is to produce a 3D structural movie of dynamic macromolecules in action, providing unprecedented insights into their functional mechanisms. 
To demonstrate the utility of this proposed method, we have applied it to investigate the dynamic structure of \ac{RNAP} during promoter escape.

In summary, this work not only contributes to advancing our understanding of the intricate dynamics of macromolecular machines but also paves the way for a new era of structural biology, where real-time observations of complex biological processes at the atomic scale become a reality.
   