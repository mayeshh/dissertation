\chapter{48-spot setup and parts}
\label{chpt:setup_appendix}

The make and model of the parts used in the 48-spot setup are included here for researchers interested in constructing a multi-excitation wavelength, multispot microscope based on \ac{LCOS-SLM}s. 

In Figure~\ref{fig:setup}, you can find a schematic representation of the 48-spot setup. 
This setup employs two \ac{CW} lasers, each with a power of 1~W. 
These lasers, the 2RU-VFL-Series from MPB Communications, Inc. in Quebec, Canada, emit light at two different excitation wavelengths: 532~nm (green) and 628~nm (red). 
The intensity of both lasers can be controlled either through software or by adjusting a polarizer.

For the red laser, an \ac{AOM} (P/N 48058 PCAOM, with corresponding electronics: P/N 64048-80-.1-4CH-5M, Neos Technology, Melbourne, FL) was used. 
The \ac{AOM} was driven by a square wave (\ac{TTL}) with a period of 51.2~$mu$s and a 50\% duty cycle. 
To align the polarization with the expected orientation at the \ac{LCOS-SLM}s, each laser's polarization was individually adjusted using a half-wave plate.

Initially, both laser beams are expanded and collimated through a pair of doublet lenses, forming a Keplerian telescope, with focal lengths of $f_1 = 50$~mm and $f_2 = 250$~mm (not depicted in Fig.~\ref{fig:setup}). 
Following this, the laser beams are directed toward the optical breadboard that the microscope sits on. 
This is accomplished using two periscopes, and the beams are additionally expanded by two adjustable beam expanders ($BE_{G}$ and $BE_{R}$: 3X, P/N 59-131, Edmund Optics).

Subsequently, each expanded beam is directed towards its respective \ac{LCOS-SLM}, with the green laser directed to \ac{LCOS-SLM} P/N X10468-01 (Hamamatsu, Japan) and the red laser to \ac{LCOS-SLM} P/N X10468-07. 
The \ac{LCOS-SLM}s generate an array of spots at their focal plane, as illustrated in Figure \ref{fig:LCOS_params}. 
The emitted light from these spots is initially combined using a dichroic mirror, $DM_{mix}$ (T550LPXR, Chroma Technology, VT), and then focused onto the microscope's focal plane. 
This is achieved by employing a collimating lens, $L_3$ ($f = 250$~mm, AC508-250-A Thorlabs), and a water immersion objective lens (UAPOPlan NA 1.2, 60X, Olympus). 
A dual-band dichroic mirror, $DM_{EX}$ (Brightline FF545/650-Di01, Semrock, NY), is employed to separate the excitation and emission light.

Fluorescence emission is focused by the microscope tube lens, denoted as $L_2$. 
Within the microscope, an internal flip mirror, $M_I$, serves the purpose of toggling between the side and bottom ports of the microscope. 
A \ac{CMOS} camera (Grasshopper3 GS3-U3-23S6M-C model from FLIR in BC, Canada) is affixed to the side port, primarily for alignment purposes. 
In contrast, the bottom port directs the emitted fluorescence to a recollimating lens, labeled as $L_4$ ($f = 100$~mm, AC254-100-A, Thorlabs). 
The light is subsequently split using an emission dichroic mirror, $DM_{EM}$ (Brightline Di02-R635, Semrock). 
To mitigate spectral leakage from the red laser and combat Raman scattering associated with the green laser, additional band-pass filters are incorporated into the donor emission path (donor: Brightline FF01-582/75, Semrock).

Lens $L_{5}$ ($f = 150$ mm, AC254-150-A, Thorlabs) serves to focus each signal onto its respective \ac{SPAD} array. 
The \ac{SPAD} arrays are situated on micro-positioning stages, providing the capability to make adjustments in all three dimensions. 
For precise alignment in the x and y directions, open-loop piezo-actuators (P/N 8302; drivers: P/N 8752 and 8753; Newport Corporation, Irvine, CA) are employed and controlled via software.

Each SPAD array is outfitted with several components: an \ac{FPGA} based on the Xilinx Spartan 6 model SLX150, a humidity sensor, and a USB connection for monitoring time-binned counts and humidity levels. 
The \ac{FPGA} provides 48 parallel and independent streams of \ac{LVDS} pulses. 
These \ac{LVDS} pulses are then converted to \ac{TTL} pulses before being routed to a programmable counting board (PXI-7813R model from National Instruments in Austin, TX). 
This board provides 12.5~ns resolution for time-stamping and assigns a unique channel ID to each pulse. 
The LabVIEW code that programs the \ac{FPGA} module is accessible in the \texttt{Multichannel-Timestamper} online repository~\href{https://github.com/multispot-software/MultichannelTimestamper}{repository}.

The specific acquisition board employed in this work is no longer in production, but it may still be obtainable from third-party vendors. 
As an alternative, consider one of the PXI-78XYR boards (where X = 3, 4, or 5, and Y = 1, 2, 3, or 4) which offer 96 digital inputs and feature higher-performance \ac{FPGA}s.